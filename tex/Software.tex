\section{Software}
\label{sec:software}

Softwareseitig handelt es sich um eine Eigenentwicklung, den DWRasp-Rail-Controller (DWRRC).
Dieser setzt auf einer weiteren Eigenentwicklung auf, n\"amlich dem Projekt DWRasp-on-PsiCore.
Die Git-Projekte zu beiden Projekten sind unter folgenden Links verf\"ugbar:
\begin{itemize}
	\item \textit{https://github.com/dwoelki/DWRasp-on-PsiCore}
	\item \textit{https://github.com/dwoelki/DWRasp-Rail-Controller}
\end{itemize}
Zu beachten ist, dass das DWRasp-on-PsiCore Projekt selbst Drittsoftware einbindet.
Dabei handelt es sich um das $\Psi$-Core, das am Institut f\"ur Luft- und Raumfahrt der Technischen Universit\"at Berlin (ILR) entwickelt wurde.
Das $\Psi$-Core ist die Engine der ILR-hauseigenen Simulationsumgebung IPSM (Interface for Performance and Secondary Air System Modeling) \cite{Woe14}, \cite{Woe19c}.
Zuf\"alligerweise sind der Sch\"opfer bzw. Hauptentwickler von IPSM sowie der Sch\"opfer von Granitz ein und dieselbe Person.

Der Quellcode aller genannten Projekte ist fast vollst\"andig in der Programmiersprache Java geschrieben und objektorientiert.
Das $\Psi$-Core sowie alle darauf aufbauenden Anwendungen, die ich schreibe (also auch DWRRC) sind \"ublicherweise stark modularisiert.
Im vorliegenden Fall ist die Modularisierung aber linear und somit recht leicht nachvollziehbar.
Als bevorzugte IDE wurde Netbeans IDE (sowohl in vor- als auch zu-Apache Zeiten) genutzt.
Die aus den Projekten erzeugten Bibliotheken entsprechen somit \"ublicherweise Modulen/Plugins mit entsprechenden Netbeans Spezifikationen.
Am einfachsten ist daher die Adaption der Projekte direkt mit Netbeans als frei verf\"ugbare IDE.
Eine Konvertierung auf andere IDEs bzw. Extrahierung des ben\"otigten Quellcodes bedeutet f\"ur den ge\"ubten Programmierer aber nat\"urlich kein gro{\ss}er Aufwand.





\subsection{Hardwareansteuerung \"uber GPIO}
\label{sec:gpio}



\subsection{Remotesteuerung \"uber IRC}
\label{sec:remote_irc}