\section*{Abstract}
\label{sec:abstract}

Dieses Dokument beschreibt die Hintergrund- und Entstehungsgeschichte der privaten Modelleisenbahnanlage Granitz.
Das Thema der Anlage ist die fiktive Stadt Granitz, die in Brandenburg, tendenziell s\"ud\"oestlich von Berlin lokalisiert ist.
Granitz liegt an einer zweigleisigen Hauptsrecke, die einen gut frequentierten Regionalverkehr sowie einige Fernverkehrsverbindungen anbietet.
Die Stadt ist im RE und RB Betrieb routinem\"a{\ss}iger Haltepunkt zwischen Berlin und dem etwa gleich gro{\ss}en fiktiven Ort Schattenwalde.

Das Szenario ist zeitlich in Epoche V verordnet.
Hier nimmt es aber auch die komplette Bandbreite ein:
War Granitz um die Wende herum noch eine belebte Stadt mit angesiedelter Industrie und regem G\"uterverkehraufkommen im St\"uckverkehr, so ist es in den 2000ern vergessen worden.
Erkl\"artes Ziel der Anlage ist es daher auch, beide Zust\"ande abzubilden, sowohl betrieblich als auch im Diorama.
Letzteres soll relativ einfach durch Austausch weniger, gestalterischer Anlagenelemente erfolgen.

Die komplette Anlage ist auf Normalspur in Spur N, Ma{\ss}stab \textit{1:160} ausgef\"uhrt.
Bei dem Gleissystem handelt es sich um das Stecksystem Arnold N.
Der Betrieb ist analog.
Eine computergest\"utzte Blocksteuerung und somit weitestgehende \"uber PC and ggf. Smartphone einstellbare Teilautomatisierung ist vorgesehen.