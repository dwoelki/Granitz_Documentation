\section{Szenische Ausgestaltung}
\label{sec:diorama}



\subsection{Geb\"aude}

Durch die Kindheit und Platten im famili\"aren Umfeld lag die Pr\"agung tendenziell auf romantischen Motiven in den Voralpen.
Herstellerseitig lag somit ein tiefer Bezug zu den Firmen
\textit{Faller}\footnote{Faller ist ein eingetragender Markenname der Faller GmbH. Alle Angaben zum Sortiment spiegeln die pers\"onlichen Ansichten des Autors wider und sind in keinster Weise despektierlicher Natur.} und
\textit{Vollmer}\footnote{Vollmer ist ein eingetragender Markenname der Vollmer GmbH \& Co. KG i.~L., inzwischen durch die Viessmann Modelltechnik GmbH vertrieben. Alle Angaben zum Sortiment spiegeln die pers\"onlichen Ansichten des Autors wider und sind in keinster Weise despektierlicher Natur.} vor.
Leider sind die neuen Bundesl\"ander, insbesondere abseits der Mittelgebirge, in den Sortimenten nicht so gut vertreten.
Die Idee war daher, auf Modellbahnb\"orsen sammeln zu gehen und ggf. sogar selbst zu basteln.
Ein erster Besuch auf einer Bahnb\"orse in Berlin war allerdings etwas ern\"uchternd und die F\"ahigkeit zu hochpr\"zisem Basteln ist eigentlich auch nicht vorhanden.

Ein gro{\ss}er Lichtblick war das Auffinden der Firma
\textit{Auhagen}\footnote{Auhagen ist ein eingetragender Markenname der Auhagen GmbH. Alle Angaben zum Sortiment spiegeln die pers\"onlichen Ansichten des Autors wider und sind in keinster Weise despektierlicher Natur.}.
In Sachsen beiheimatet, liegt hier ein ganz deutlicher Bezug zu den neuen Bundesl\"andern vor.
Das H0-Sortiment weist eine hervorragende F\"ulle auf.
In Spur N ist es schon ein bisschen \"ubersichtlicher, aber das wichtigste ist vorhanden.



\subsubsection{Bahnhofsgeb\"aude Granitz}

Das Hauptgeb\"aude des Bahnhofs ist ein f\"ur die Region typischer, repr\"asentativer Backsteinbau.
\begin{itemize}
	\item[] Hersteller:~Auhagen
	\item[] Bezeichnung:~Bahnhof Neupreu{\ss}en
	\item[] Artikel-Nr.:~14485
\end{itemize}

\subsubsection{Bahnsteige in Granitz}

Prim\"are, \"uberdachte Bahnsteige:
\begin{itemize}
	\item[] Hersteller:~Auhagen
	\item[] Bezeichnung:~Bahnsteig
	\item[] Artikel-Nr.:~14481
\end{itemize}
Ggf. auf Bahnsteig Gleis IV/V verl\"angert um moderenen, nachger\"usteten, sekund\"aren Bahnsteig:
\begin{itemize}
	\item[] Hersteller:~Auhagen
	\item[] Bezeichnung:~Bahnsteig
	\item[] Artikel-Nr.:~14459
\end{itemize}

Nicht \"uberdachte Verl\"angerungungen durch gefertigte Bahnsteigkanten und selbstgemachter Verf\"ullung der Bahnsteigfl\"achen \hl{Holz/Pappe, Gips?}.
\begin{itemize}
	\item[] Hersteller:~Auhagen
	\item[] Bezeichnung:~Bahnsteigkanten
	\item[] Artikel-Nr.:~44631
\end{itemize}

\subsubsection{Bahnsteige in Granitz-Walddorf}

\"Uberdachte Bahnsteige:
\begin{itemize}
	\item[] Hersteller:~Auhagen
	\item[] Bezeichnung:~Bahnsteig
	\item[] Artikel-Nr.:~14459
\end{itemize}

Nicht \"uberdachte Verl\"angerungungen durch gefertigte Bahnsteigkanten und selbstgemachter Verf\"ullung der Bahnsteigfl\"achen \hl{Holz/Pappe, Gips?}.
\begin{itemize}
	\item[] Hersteller:~Auhagen
	\item[] Bezeichnung:~Bahnsteigkanten
	\item[] Artikel-Nr.:~44631
\end{itemize}



\subsubsection{Wohngeb\"aude in Granitz}

Viele B\"urger sind in Wohnplatten untegebracht:
\begin{itemize}
	\item[] Hersteller:~Auhagen
	\item[] Bezeichnung:~Mehrfamilienhaus
	\item[] Artikel-Nr.:~14472
\end{itemize}
Diese Wohnplatte eignet sich auch f\"ur die Darstellung verlassener, im Verfall befindlicher Wohn- oder Kasernenanlagen.
Aus zwei erworbenen Artikeln sind die Fassaden f\"ur zwei gesunde und ein alternatives, verlassenes Wohnobjekt verf\"ugbar.
Die Seitenwand und ein Flachdach des verlassenen Hauses kann mit \hl{Holz/lackierter Pappe} nachgebaut werden.

\begin{itemize}
	\item[] Hersteller:~
	\item[] Bezeichnung:~
	\item[] Artikel-Nr.:~
\end{itemize}

\begin{itemize}
	\item[] Hersteller:~
	\item[] Bezeichnung:~
	\item[] Artikel-Nr.:~
\end{itemize}

\begin{itemize}
	\item[] Hersteller:~
	\item[] Bezeichnung:~
	\item[] Artikel-Nr.:~
\end{itemize}