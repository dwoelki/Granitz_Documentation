\ifthenelse{\boolean{INTERNAL_VERSION}}{

	\section*{NOTE TO REVIEWERS}
	
	\begin{table}[h]
	\begin{tabular}{||l||}
	\hline
	\textbf{Highlighted text (\hl{this way}) is either meant to be for internal version only}\\
	\textbf{and to be removed for version to be published \textit{or} information to be verified.}
	\\	
	If a parameter name of a table's column is highlighted, the entire column will be removed.\\
	Same applied to a table's line, if the line's first cell is highlighted.
	\\
	If the entire caption of a figure is highlighted, the entire figure will be removed.\\
	If single sub-figure captions (within a minimap) are highlighted, only these sub-figures will be removed.
	\\
	\hline
	\end{tabular}
	\end{table}
	
}{
	
	\section*{Vorbemerkungen zur Verwendung des Dokuments}
	
	\textit{Wirklich zuallererst:} Um zu entscheiden, ob man an diesem Dokument \"uberhaupt interessiert ist, liest man sich am Besten das halbseitige \textit{Abstract} durch (quasi der Buchr\"ucken, Kap.~\ref{sec:abstract}).
	Danach kehrt man am Besten hierher zur\"uck, denn am Seitenende geht es auch um die notwendigen, rechtlichen Rahmenbedingungen etc.
	
	Diese Dokumentation ist ein Freizeitwerk und handelt ein Hobby ab, in dem der Spa{\ss} im Mittelpunkt stehen soll.
	Der Autor ist ein Dokumentationsfetischist, um Entwicklungen festzuhalten (d.h. Versionierung, auch im strengen Sinne!) und Wissen zu teilen.
	Dem Autor geht es dabei allgemeinhin nicht um Selbstdarstellung, sondern Wissenstransfer und Aufmunterung zur Zusammenarbeit.
	KollegInnen k\"onnen das i.d.R. best\"atigen.
	
	Der Autor spricht aus Gewohnheit h\"aufig von sich in der dritten Person.
	Er h\"alt sich aber nicht f\"ur Julius Caesar.
	Die personalisierte Schreibform geh\"ort einfach nicht in den beruflichen Alltag, wenn es um Dokumentation geht.
	Hier und da erinnert sich der Autor daran, so dass gelegentlich das 'Ich' in Erscheinung tritt.
	
	Text, der \hl{in dieser Art gehighlighted} ist, markiert entweder unverifizierten Inhalt, der zu einem sp\"ateren Zeitpunkt einem entsprechenden Review unterzogen wird, oder (noch) unvollst\"andigen Inhalt.
	Die Dokumentationssprache ist Deutsch.
	Sporadisches Aufflammen eines gewissen Denglisch ist wiederum dem Berufsalltag geschuldet.
	Separierte Dokumentationen von mir, die sich mit sp\"ater eingesetzter, selbst geschriebener Software oder Hardwareansteuerung befassen, sind i.d.R. in Englisch dokumentiert.
	Sorry, als Dokumentationssprache ist Englisch heutzutage einfach sinnvoller.
	Was die Modellbahn selbst betrifft, bleibe ich lieber zu Hause oder zumindest in der Region.
	Von daher ist das noch ein sinnvoller Bereicht f\"ur deutschsprachige Dokumentation.
	Au{\ss}erdem haben Amerikaner tendenziell andere Vorstellungen davon, wie ein Zug aussehen sollte, als ich.
	
	Die Angaben in dieser Dokumentation spiegeln mal mehr, mal weniger abstrakt Ansichten des Autors wider.
	Auch f\"ur den Fall, dass einige Passagen, insbesondere was Lokalpatriotismus und das regionale Umfeld des modellierten Szenarios betrifft, etwas bissig erscheinen, so sind sie niemals b\"oser Natur.
	Der Autor ist zuallererst begr\"u{\ss}ender Erdb\"urger, danach erst Berliner und Europ\"aer.
	Die Staatlichkeit spielt f\"ur ihn sehr untergeordnete Rolle.
	Der Autor begr\"u{\ss}t alle nicht menschenverachtenden Lebensmodelle.
	
	Die Dokumentation wird unentgeltlich zur Verf\"ugung gestellt und ich erhalte von niemandem daf\"ur eine Verg\"utung in welcher Form auch immer.
	Die Nennung von Marken, die im Modelleisenbahnbau g\"angig sind, sind nicht als Werbung an sich zu verstehen, sondern sollen Aufschluss \"uber verwendete Baus\"atze und Zugmaterial geben.
	Die genannten Marken sind i.d.R. eingetragene, gesch\"utzte Markennamen.
	Alle Bemerkungen zu deren Sortimenten sind rein zweckgebunden und in keinem Fall allgemein-wertend zu verstehen.
	Alle genannten Modelleisenbahnhersteller (und dar\"uber hinaus noch viel mehr) sind meiner Meinung nach per se einem gewissen Idealismus verschrieben, so dass ihre Daseinsberechtigung und ihre Sortimente unsere Hobbywelt bereichern.
	
	Die Dokumentation wird zur Verf\"ugung gestellt, wie sie als jeweiliges PDF-Dokument kompiliert ist (\textit{'as is'}).
	Die Selbstkompilierung oder anderweitige Nutzung von Rohdateien erfolgt auf eigene Verantwortung.
	Dies betrifft den vollst\"andigen Umfang des Git-Projekts, das unter folgendem Link abrufbar ist:
	\begin{itemize}
		\item[] \textit{https://github.com/dwoelki/Granitz\_Documentation}
	\end{itemize}
	Der Autor \"ubernimmt keine Haftung f\"ur Sch\"aden, die durch die Dokumentation \textit{as is} oder Folgesch\"aden, z.B. durch Nachbau, entstehen.
	Der Autor \"ubernimmt generell keine Haftung und Garantie f\"ur irgendwelche Teile dieser Dokumentation.
	Verlinkte Inhalte werden nach bestem Wissen und Gewissen auf Legalit\"at und politische Korrektheit \"uberpr\"uft.
	Gleichwohl \"ubernimmt der Autor keine Haftung f\"ur verlinkte Inhalte.
	
	Das Dokument sowie das gesamte Git-Repository mit Quell- und Erg\"anzungsdateien sind unter der \textit{Apache 2.0 License} lizensiert \cite{Apa04}.
	Der vollst\"andige Text der Lizenz als Template, zu erg\"anzen um den Namen des Autors, ist verf\"ugbar unter folgendem Link:
	\begin{itemize}
		\item[] \textit{https://www.apache.org/licenses/LICENSE-2.0.txt}
	\end{itemize}
	
	Dar\"uber hinaus gelten alle Konditionen der Apache 2.0 License.
	
	\vspace{3cm}
	
	\raggedright{Berlin, den 24. Mai 2020}
	
}